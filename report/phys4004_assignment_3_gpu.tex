\documentclass{article}

% - Style
\usepackage{base}

% - Plotting
\usepackage{pgfplotstable}
\usepgfplotslibrary{units}
\usetikzlibrary{external}
\tikzexternalize[mode=list and make]

% - Listings
\usepackage{color}
\usepackage{listings}

\lstset{
  basicstyle=\ttfamily\footnotesize\color{black}
  , commentstyle=\color{blue}
  , keywordstyle=\color{purple}
  , stringstyle=\color{orange}
  %
  , numbers=left
  , numbersep=5pt
  , stepnumber=1
  , numberstyle=\ttfamily\small\color{black}
  %
  , keepspaces=true
  , showspaces=false
  , showstringspaces=false
  , showtabs=false
  , tabsize=2
  , breaklines=true
  %
  , frame=single
  , backgroundcolor=\color{white}
  , rulecolor=\color{black}
  , captionpos=b
  %
  , language=C
}

% file or folder
\lstdefinestyle{ff}{
  basicstyle=\ttfamily\normalsize\color{orange}
}

\newcommand{\lilf}[1]{\lstinline[style=ff]{#1}}

% - Title
\title{PHYS4004 - Assignment 3 - Game of Life (GPU)}
\author{Tom Ross - 1834 2884}
\date{}

% - Headers
\pagestyle{fancy}
\fancyhf{}
\rhead{\theauthor}
\chead{}
\lhead{\thetitle}
\rfoot{\thepage}
\cfoot{}
\lfoot{}

% - Document
\begin{document}

\tableofcontents

\clearpage
\section{Overview}
\label{sec:overview}

The entire code repository can be found at
\url{https://github.com/dgsaf/game-of-life-gpu}.
Conway's Game of Life has been accelerated using GPU programming with two
models, CUDA and OpenACC, both using C.
The code has been derived from the original code which was provided by Cristian
Di Pietrantonio, and Maciej Cytowski.
It consists of the following items of interest:
\begin{itemize}
\item \lilf{report/}:
  The directory containing this \lilf{tex} and its resulting \lilf{pdf}
  document.

\item \lilf{ex1-gol-cuda/}, \lilf{ex2-gol-gpu-directives/openacc/}:
  The CUDA and OpenACC directories have a similar structure, which consists of:

  \begin{itemize}
  \item \lilf{cpu.slurm}:
    A \lilf{slurm} script for submitting CPU jobs on Topaz, for given
    \lstinline{n, m, nsteps}.
    The CPU code timing output, recorded in [\si{\milli\second}], is written to
    \lstinline{output/timing-cpu.n-<n>.m-<m>.nsteps-<nsteps>.txt}.

  \item \lilf{gpu.slurm}:
    A \lilf{slurm} script for submitting GPU CUDA jobs on Topaz, for given
    \lstinline{n, m, nsteps}.
    The GPU code timing output, recorded in [\si{\milli\second}], is written to
    \lilf{output/timing-gpu-cuda.n-<n>.m-<m>.nsteps-<nsteps>.txt} for the CUDA
    code, and \lilf{output/timing-gpu-openacc.n-<n>.m-<m>.nsteps-<nsteps>.txt}
    for the OpenACC code.

  \item \lilf{jobs.sh}:
    A \lilf{bash} script which batches a set of jobs, for both
    the CPU and the GPU codes, on Topaz, for \lstinline{nsteps = 100} and
    \lstinline{n = m = 1, 2, 4, 8, ..., 16384}.

  \item \lilf{extract.sh}:
    A \lilf{bash} script which, from the jobs batched in \lilf{jobs.sh},
    \lstinline{n = m = 1, 2, 4, 8, ..., 16384}, extracts the timing output
    \lstinline{cpu_elapsed_time, cpu_elapsed_time, kernel_time}, calculates
    \lstinline{speedup}, and writes this performance evaluation to
    \lilf{output/performance.nsteps-<nsteps>.txt}.

  \item \lilf{output/performance.txt}:
    A \lilf{txt} file which, after the jobs have been submitted and the timing
    output extracted, contains for each job
    \lstinline{n = m = 1, 2, 4, 8, ..., 16384} the performance characteristics
    \lstinline{cpu_elapsed_time, cpu_elapsed_time, speedup, kernel_time}.
  \end{itemize}
\end{itemize}

\section{CPU Code}
\label{sec:cpu-code}

The original code for both the CUDA and OpenACC models has been modified
slightly.
\begin{itemize}
\item
  Minor C formatting changes have been made, although only where the original
  code was modified - unmodified regions of the code remain unadjusted.

\item
  Debugging macros have been utilised to annotate the code for clarity, and can
  be compiled away to yield performant code.

\item
  The timing methods in \lilf{common.c}, \lilf{common.h}, have been
  standardised across the CUDA and OpenACC codes; having originally yielding
  different return types in each model.
  The function \lstinline{float get_elapsed_time(struct timeval start)} now
  returns the time since \lstinline{start} was initialised, in
  [\si{\milli\second}], for both codes.

\item
  In \lilf{common.c}, \lilf{common.h}, the ASCII visualisation has been modified
  in the following ways: it truncates the grid so that even large grids can be
  partially visualised in the terminal - allowing for easier verification of
  grid states across codes, it builds the visualisation output in a buffer
  string which is the printed in one call to avoid interference from
  asynchronous terminal behaviour.

\end{itemize}

It should be noted that the debugging macros are also ported to the GPU codes,
and so, are common to the CPU and GPU codes across both models.
The debugging macros are shown in \autoref{lst:debug_macros}, and can be called
with a format string, and variable number of arguments similar to how
\lstinline{printf()} is called.

\lstinputlisting[
language=C
, linerange={3-31}
, firstnumber=3
, caption={
  Debugging macros from \lilf{ex1-gol-cuda/src/game_of_life.c}.
  Note that these macros are common to both CPU and GPU codes across both
  models.
}
, label={lst:debug_macros}
]{../ex1-gol-cuda/src/game_of_life.c}

\section{GPU Code}
\label{sec:Gpu-code}

\subsection{CUDA Code}
\label{sec:cuda-code}

% figure - performance

\subsection{OpenACC Code}
\label{sec:cpu-code}

% figure - performance

\end{document}