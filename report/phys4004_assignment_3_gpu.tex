\documentclass[draft]{article}

% - Style
\usepackage{base}

% - Plotting
\usepackage{pgfplotstable}
\usepgfplotslibrary{units}
\usetikzlibrary{external}
\tikzexternalize[mode=list and make]

% - Listings
\usepackage{color}
\usepackage{listings}

\lstset{
  basicstyle=\ttfamily\footnotesize\color{black}
  , commentstyle=\color{blue}
  , keywordstyle=\color{purple}
  , stringstyle=\color{orange}
  %
  , numbers=left
  , numbersep=5pt
  , stepnumber=1
  , numberstyle=\ttfamily\small\color{black}
  %
  , keepspaces=true
  , showspaces=false
  , showstringspaces=false
  , showtabs=false
  , tabsize=2
  , breaklines=true
  %
  , frame=single
  , backgroundcolor=\color{white}
  , rulecolor=\color{black}
  , captionpos=b
}

% file or folder
\lstdefinestyle{ff}{
  basicstyle=\ttfamily\normalsize\color{orange}
}

\newcommand{\lilf}[1]{\lstinline[style=ff]{#1}}

% - Title
\title{PHYS4004 - Assignment 3 - Game of Life (GPU)}
\author{Tom Ross - 1834 2884}
\date{}

% - Headers
\pagestyle{fancy}
\fancyhf{}
\rhead{\theauthor}
\chead{}
\lhead{\thetitle}
\rfoot{\thepage}
\cfoot{}
\lfoot{}

% - Document
\begin{document}

\tableofcontents

\clearpage
\section{Overview}
\label{sec:overview}

The entire code repository can be found at
\url{https://github.com/dgsaf/game-of-life-gpu}.
Conway's Game of Life has been accelerated using GPU programming with two
schemes, CUDA and OpenACC.
The code has been derived from the original code which was provided by Cristian
Di Pietrantonio, and Maciej Cytowski.
It consists of the following items of interest:
\begin{itemize}
\item \lilf{report/}:
  The directory containing this \lilf{tex} and its resulting \lilf{pdf}
  document.

\item \lilf{ex1-gol-cuda/}, \lilf{ex2-gol-gpu-directives/openacc/}:
  Both the CUDA and OpenACC directories have a similar structure, which consists
  of:

  \begin{itemize}
  \item \lilf{cpu.slurm}:
    A \lilf{slurm} script for submitting CPU jobs on Topaz, for given
    \lstinline{n, m, nsteps}.
    The CPU code timing output, recorded in [\si{\milli\second}], is written to
    \lstinline{output/timing-cpu.n-<n>.m-<m>.nsteps-<nsteps>.txt}.

  \item \lilf{gpu.slurm}:
    A \lilf{slurm} script for submitting GPU CUDA jobs on Topaz, for given
    \lstinline{n, m, nsteps}.
    The GPU code timing output, recorded in [\si{\milli\second}], is written to
    \lilf{output/timing-gpu-cuda.n-<n>.m-<m>.nsteps-<nsteps>.txt} for the CUDA
    code, and \lilf{output/timing-gpu-openacc.n-<n>.m-<m>.nsteps-<nsteps>.txt}
    for the OpenACC code.

  \item \lilf{jobs.sh}:
    A \lilf{bash} script which batches a set of jobs, for both
    the CPU and the GPU codes, on Topaz, for \lstinline{nsteps = 100} and
    \lstinline{n = m = 1, 2, 4, 8, ..., 16384}.

  \item \lilf{extract.sh}:
    A \lilf{bash} script which, from the jobs batched in \lilf{jobs.sh},
    \lstinline{n = m = 1, 2, 4, 8, ..., 16384}, extracts the timing output
    \lstinline{cpu_elapsed_time, cpu_elapsed_time, kernel_time}, calculates
    \lstinline{speedup}, and writes this performance evaluation to
    \lilf{output/performance.nsteps-<nsteps>.txt}.

  \item \lilf{output/performance.txt}:
    A \lilf{txt} file which, after the jobs have been submitted and the timing
    output extracted, contains for each job
    \lstinline{n = m = 1, 2, 4, 8, ..., 16384} the performance characteristics
    \lstinline{cpu_elapsed_time, cpu_elapsed_time, speedup, kernel_time}.
  \end{itemize}
\end{itemize}

\section{CPU Code}
\label{sec:cpu-code}

\section{GPU Code}
\label{sec:Gpu-code}

\subsection{CUDA Code}
\label{sec:cuda-code}

% figure - performance

\subsection{OpenACC Code}
\label{sec:cpu-code}

% figure - performance

\end{document}